%%%%%%%%%%%%%%%%%%%%%%%%%%%%%%%%%%%% Sistemas Operativos y Persistencia
%%%%%%%%%%%%%%%%%%%%%% jbgarcia@uvigo.es
\documentclass{article}

\title{Sistemas Operativos y Persistencia}
\author{Garc\'{\i}a Perez-Schofield J. Baltasar.
   \and Departamento de Inform\'atica. Universidad de Vigo, Espa\~na(Spain).
   \and jbgarcia@uvigo.es
}



\begin{document}
\maketitle

%%%%%%%%%%%%%%%%%%%%%%%%%%%%%% Abstract %%%%%%%%%%%%%%%%%%%%%%%%%%%%%%%%%%%%%%

\begin{abstract}
Los sistemas de programaci\'on persistentes realizan autom\'aticamente la labor de almacenamiento y
recuperaci\'on de las estructuras de datos de las aplicaciones. Las ventajas de esta automatizaci\'on son
principalmente tres: el eliminar una parte de la aplicaci\'on que es tediosa de programar, repetitiva y
que supone un tama\~no considerable de l\'ineas de c\'odigo; eliminar los errores derivados de la
programaci\'on de las partes de la aplicaci\'on anteriores; y, finalmente, eliminar la diferenciaci\'on
entre memoria principal y secundario, siendo un todo cont\'inuo para el programador.
Los sistemas operativos de la actualidad se basan en el dise\~no del sistema operativo UNIX,
creado en los a\~nos 70. Su filosof\'ia de trabajo se basa en la met\'afora de ficheros: cualquier elemento
del sistema operativo (driver, proceso en memoria ... etc), puede ser tratado como un fichero, de
forma que se puede abrir, leer, y cerrar. En este sentido, el estancamiento en sistemas operativos es
patente. El avance en este campo ha venido de la mano de los interfaces de usuario: \'estos permiten
obtener una visi\'on del computador orientada a objetos hasta, un cierto grado, ya que el hecho de
estar basados en un sistema operativo tradicional supone varias limitaciones. En este coloquio se
examina esta situaci\'on en detalle, y se propone una soluci\'on basada en sistemas operativos
orientados a objetos.
\end{abstract}


%%%%%%%%%%%%%%%%%%%%%%%%%%%%%% Introduci\'on %%%%%%%%%%%%%%%%%%%%%%%%%%%%%%%%%%%
\section{Introducci\'on}
"Persistencia" es un concepto asociado a la orientaci\'on a objetos: se trata, ni m\'as, ni menos,
de poder almacenar y recuperar objetos, de igual forma que se puede almacenar y recuperar
un entero, una cadena, o un flotante. 

Por otra parte, el sistema operativo, como es bien sabido, es el software m\'inimo que
precisa un ordenador para poder funcionar, ofreciendo las posibles funcionalidades
del hardware al usuario. 


En este art\'iculo, se presenta la evoluci\'on de ambos conceptos, y la relaci\'on entre ellos.


\subsection{Persistencia}
El grado de transparencia que un sistema orientado a objetos
debe proporcionar para que su mecanismo de salvaguarda y recuperaci\'on de objetos sea
considerado como persistente var\'ia de un sistema a otro y de un autor a otro.
Sin embargo, se acepta globalmente un tipo de persistencia especial, la 
"Persistencia Ortogonal", (ortogonalidad en el sentido de independencia)
que se basa en tres principios (\cite{Atkinson&Morrison, 1995}):
\begin{itemize}
\item \textbf{Independencia del tipo:} Un objeto puede ser almacenado y recuperado sin importar
de qu\'e tipo o clase es.
\item \textbf{Independencia del trato:} No deber\'ian existir diferencias en el trato entre objetos
persistentes y no persistentes.
\item \textbf{Transitividad de persistencia:} La decisi\'on de si un objeto debe ser o no persistente
corresponde al sistema, no al programador, y normalmente \'esto se consigue mediante
alg\'un algoritmo de referencia desde una ra\'iz persistente.
\end{itemize}
Lo que ha llegado a los lenguajes de programaci\'on orientados a objetos, de persistencia,
es m\'as bien poco. Mientras que C++ incorpora el mismo mecanismo \cite{Garc�a Perez-Schofield&P�rez Cota, 1998}
que ya ten\'ia C (\verb"fwrite(f, &Obj, sizeof(obj), 1);"), 
Java avanza un poco m\'as e incorpora un mecanismo
de serializaci\'on m\'as avanzado (\cite{Naughton, 1996}), que permite serializar el cierre persistente 
(\cite{Atkinson&Morrison, 1995})(es decir,
todos los objetos relacionados con uno dado).
\subsection{Sistemas Operativos}
Los sistemas operativos han evolucionado sobre todo desde la aparici\'on del primer
sistema operativo "popular", el sistema operativo UNIX (Copyright SCO/Caldera, inventado en Laboratorios Bell, 1969). 
Varios sistemas operativos
aparecieron m\'as o menos m\'as tarde, como CP/M, MS-DOS, Windows, 
Solaris(Copyright Digital Research y Microsoft, respectivamente.), etc.

Si bien al principio pod\'ia decirse que UNIX era para ordenadores "grandes" (\textit{mainframes}
de 16 \'o 32 bits), y que CP/M y MS-DOS (y posteriormente, Windows), eran para ordenadores
"peque\~nos" (de 8 y 16 bits), \'esta divisi\'on ya no existe, puesto que los ordenadores
considerados "peque\~nos" han crecido hasta igualar y superar en prestaciones a los "grandes".
\'esto conlleva que los sistemas operativos han tenido que "crecer" a su vez en complejidad
y en capacidad (si bien \'esto ha conllevado a su vez la desaparici\'on de varios
sistemas operativos, como CP/M, MS-DOS, y la rama Windows9x de Windows, cuyo \'unico representante
es ya Windows Millenium.)

Pero por si esto fuera poco, al igualarse la capacidad de los ordenadores a los \textit{mainframes},
ha sido posible portar los sistemas operativos de \'estos \'ultimos a los primeros, dando como
resultado "nuevos" sistemas operativos (como Linux o Solaris para PC).

Y \'esto constituye todo atisbo de evoluci\'on en los sistemas operativos desde UNIX (\cite{Balch et al., 1989}).








%%%%%%%%%%%%%%%%%%%%%%%%%%%%% Sistemas Operativos %%%%%%%%%%%%%%%%%%%%%%%%%%%%%%%%%%%
\section{Sistemas Operativos}
\subsection{Historia de los Sistemas Operativos Windows y Linux}
La evoluci\'on de los sistemas operativos empieza, sin duda, con la creaci\'on de Unix en los
laboratorios AT\&T. Este sistema operativo estaba dirigido a \textit{mainframes}, y por
tanto, los ordenadores de consumo de la \'epoca (los microordenadores), no pod\'ian
ejecutarlo, teniendo otros sistemas operativos, como eran CP/M y MS-DOS.
El primero se escribi\'o en 1973 por Gary Killdall, para ser finalmente desarrollado
completamente por \'el mismo, ya  como Digital Research, en 1977.

El segundo es posterior (y algunos apuntan que sospechosamente parecido \dots), 
apareciendo en 1981.

Ambos eran sistemas operativos que imitaban UNIX en la medida de lo posible, si bien ambos
eran monotarea y monousuario.

\subsubsection{Microsoft Windows}
En 1985 aparece en el mercado Windows 1.0, una \textit{shell} para MS-DOS. 
Conociendo los problemas que Digital Research hab\'ia tenido con Apple, debido a GEM,
Microsoft vari\'o ligeramente el aspecto de su sistema.
En 1990 aparece Windows 3.0, que ya presume de multitarea. Las posibilidades de multitarea
de Windows 3.0 deber\'ian catalogarse m\'as bien como multiprogramaci\'on, puesto que Windows
aprovechaba el nuevo vector de interrupci\'on que MS-DOS ejecutaba cuando se quedaba esperando
por una operaci\'on de entrada/salida (el PC ten\'ia chip de DMA desde el principio, si
bien las primeras versiones de DOS eran incapaces de sacarle partido, por lo que entraban
en un bucle infinito cada vez que un programa se interrump\'ia con una operaci\'on de entrada/salida,
esperando que el vector de interrupci\'on del chip DMA le despertase).

En 1993, desarrollado desde cero, y emulando la forma de trabajar de UNIX de una forma
m\'as que evidente (exceptuando que el gestor de ventanas est\'a permanentemente cargado),
aparece Windows NT, que se presenta dividido en "cliente" (Windows NT Workstation y "servidor"
Windows NT Server).

En 1995 Microsoft comercializa Windows 95, ya con multitarea real, dentro de las
posibilidades de la arquitectura de este sistema operativo. Ya dejaba de ser una \textit{shell}
de MS-DOS, si bien necesitaba de \'este para arrancar.

Entre 2000 y 2001 aparecen Windows 98, Windows NT 4.0, y finalmente, se anuncia el abandono
de la l\'inea de Windows 95 (cuyo \'ultimo representante es Windows Me), y el establecimiento
de Windows NT como \'unico sistema operativo. As\'i, se comercializa Windows XP Professional,
y Windows XP Home Edition.
\subsubsection{GNU Linux}

En 1984, Richard Stallman abandona el Massachusets Institute of Technology, y dedica su tiempo
a desarrollar el software GNU (siglas recursivas que significan "GNU is not Unix").
Se trata de versiones abiertas (cualquiera puede obtener el c\'odigo) de aplicaciones
como emacs, vi, sh, etc.

En 1991, Linus Torvalds escribe un kernel para PC, bautizado como Linux, con el objetivo de poder trabajar con
el sistema Minix en su propio ordenador.

De la fusi\'on de ambos proyectos nace GNU Linux, un sistema operativo fuertemente basado
en Unix. Minix es una versi\'on de Unix para PC, menos potente que el Linux actual, mientras
que la colecci\'on de programas GNU no es m\'as que la colecci\'on de aplicaciones "est\'andar" que
rodean al kernel de Unix.

Actualmente, pueden encontrarse en el mercado varias distribuciones, Mandrake, Red Hat,
SuSE ...
\subsection{Hitos en la evoluci\'on de los Sistemas Operativos}
UNIX es sin duda el gran "padre" de los sistemas operativos: todos los sistemas operativos
\textbf{comerciales} de la actualidad est\'an basados en su filosof\'ia de trabajo: la
\textit{met\'afora de ficheros}.

Este concepto quiz\'as es tan habitual que pasa desapercibido en el modo de trabajo diario:
se trata de representar todos los conceptos incluidos en el sistema operativo (memoria,
proceso, gestor de dispositivo ...) como especializaciones del concepto de fichero.

As\'i, por ejemplo, en UNIX y Linux, todos los procesos activos se encuentran 
en la carpeta /proc, mientras que los gestores de dispositivos est\'an en la carpeta /dev. 
En MS-DOS, se puede escribir directamente en la impresora con un comando como 
\verb"echo hola > PRN", mientras que en Windows la llamada del
API \verb"CreateFile()" permite abrir
archivos, particiones del disco duro u otros dispositivos.

El segundo hito en la historia de los sistemas operativos es SmallTalk, desarrollado
a principios de los a\~nos 70. SmallTalk introduc\'ia varias ideas novedosas, como la
programci\'on orientada a objetos, enfocada desde un punto de vista de pureza total,
y una interfaz de usuario novedosa. Se trataba de disponer las diferentes tareas en
ejecuci\'on como ventanas, de forma que actuaban como papeles colocados encima de un
escritorio. As\'i, exist\'ia un escritorio con tal nombre e incluso una papelera.
Esta idea de interfaz de usuario ser\'ia empleada por Apple para el interfaz de usuario
de su sistema operativo, MacOS,
y posteriormente por otros como Digital Research con Gem, y Microsoft con Windows.

A pesar de la pronta aparici\'on de estos avances t\'ecnicos, 30 a\~nos despu\'es, los sistemas
operativos contin\'uan siendo b\'asicamente los mismos, es decir, derivados arquitecturalmente
de Unix.

Sin embargo, es en la interfaz de usuario donde ha aparecido la evoluci\'on m\'as interesante,
ofreciendo al usuario una \textit{vista} orientada a objetos de los ficheros a los que
accede el ordenador. As\'i, utilizando alg\'un rasgo de estos ficheros, el sistema los trata
de forma distinta. En Linux, este "rasgo" distintivo es un n\'umero llamado "MAGIC" en la 
cabecera del archivo, mientras que en windows se trata simplemente de su
extensi\'on (de ah\'i que Windows muestre un mensaje de alerta cada vez que se le
cambia la extensi\'on a un archivo).

De esta forma, pulsando con el bot\'on derecho del rat\'on, el men\'u de acciones que se le presenta
al usuario es distinto dependiendo de si el archivo contiene una imagen o un documento
de texto.

%%%%%%%%%%%%%% Sistemas Operativos Orientados a Objetos %%%%%%%%%%%%%%%%%%%%%%
\subsection{Sistemas Operativos Orientados a Objetos}

Ambos conceptos (persistencia y sistemas operativos), pueden fusionarse en uno
solo: en el \textit{SOOO (Sistema Operativo Orientado a Objetos)}. Dado que en lugar
de emplear ficheros, como en un sistema operativo tradicional, se emplear\'an
objetos, \'estos objetos deber\'an de tener, por necesidad, alguna forma de ser
almacenados y recuperados. Este mecanismo ser\'a extensible a los objetos
que el usuario pueda crear o modificar.

Existen en la actualidad varios SOOO, pero su uso es, al menos por el
momento, totalmente experimental. Por ejemplo, EROS (\cite{Shapiro et al., 1996}), basa su mecanismo
de persistencia en proceso de \textit{checkpointing}, esto es, la realizaci\'on
de volcados de memoria a disco cada determinado tiempo. Para que sea factible
la utilizaci\'on de este mecanismo, es necesario que el volcado sea tan s\'olo
incremental, es decir, que se guarden s\'olo aquellos espacios de memoria que
han sido modificados desd eel \'ultimo \textit{checkpoint}. As\'i, el \'ultimo
\textit{checkpoint} sirve de punto de referencia para reiniciar el sistema
en su \'ultimo momento estable.

Los SOOO no se han impuesto, es cierto, debido a que, principalmente, es
dif\'icil obtener el mismo rendimiento con un SOOO que con un sistema operativo
tradicional. Sin embargo, existen otras razones:

\begin{itemize}
\item \textbf{Conservadurismo: cambio de filosof\'ia de trabajo, de ficheros a objetos.}
Por alguna raz\'on, la persistencia es dif\'icil de entender para los programadores.
Probablemente sea debido a que el esquema de trabajo: "entrada-proceso-salida"
se haya visto reducido notablemente en sus partes primera y tercera.
\item \textbf{Compatibilidad hacia atr\'as nula:} Repentinamente, los sistemas operativos
ser\'ian incompatibles con todos los anteriores. Ser\'ia necesario, en todo caso,
crear alguna forma de conversi\'on de los programas y datos antiguos a los nuevos,
puesto que de otra forma ser\'ia inadmisible.
\item \textbf{Dificultades inherentes a los sistemas operativos:} sobre todo en
lo relativo a inicializaci\'on (\textit{bootstrapping}), y a manejo de
dispositivos, que hacen m\'as simple el apoyarse en versiones del sistema operativo
anteriores.
\end{itemize}


%%%%%%%%%%% El modelo de persistencia basado en contenedores %%%%%%%%%%%%%%%%

\section{El modelo de persistencia basado en contenedores}
Si bien el modelo de persistencia m\'as utilizado es el ortogonal, existen
alternativas a \'el (por ejemplo \cite{Shapiro et al., 1997}). El autor propone, a su vez, otro modelo,
el de persistencia basada en contenedores (\cite{Garc�a Perez-Schofield, 2002}).

Un contenedor no es m\'as que un objeto de gran grano, u objeto complejo; un objeto
que, en definitiva, alberga en su interior a otros objetos, sea mediante
adici\'on directa o mediante punteros.

\subsection{Clustering}
El \textit{clustering} o particionamiento es la forma de agrupar objetos a la hora de
almacenarlos en el disco duro. Existen m\'ultiples formas de \textit{clustering},
incluyendo las clases y su objetos asociados (\cite{Garc�a Perez-Schofield, 2002}) o incluso los objetos y
las clases relacionadas por el momento de su creaci\'on.

En el caso del modelo ortogonal, el clustering est\'a en todo momento oculto al
programador, debido a las fuertes restricciones que padece (en favor de una
transparencia completa). As\'i, el estudio de \textit{clustering} en este campo
se centra sobre todo en automatizaciones del mismo, e incluso readaptaciones
din\'amicas (\cite{Darmon et al., 2000}).

En el caso del modelo basado en contenedores, el particionamiento del almacenamiento
persistente es llevado a cabo por el mismo usuario (\cite{Garc�a Perez-Schofield, 2002}). Al seguir una met\'afora
basada en directorios, es posible delegar esta funci\'on en el usuario, que,
sin saberlo, particiona el almacenamiento persistente como se particiona
un disco duro o cualquier dispositivo.

Esto impide considerar a este modelo como inclu\'ido o como una especializaci\'on
del modelo ortogonal, puesto que el usuario, en definitiva, debe indicar d\'onde
guardar el objeto (al entrar en un directorio/contenedor), lo que rompe la 
ortogonalidad al trato; as\'i mismo, si bien dentro de un contenedor el sistema
realiza recolecci\'on de basura (sin perjucicio del
sistema de operadores propio de C++ \verb"new(), delete()").
de C++ \cite{Ellis & Stroustrup, 1990}), cada contenedor debe ser eliminado, en su caso, por el usuario,
lo cual rompe la tercera norma de transitividad de la persistencia.

\subsection{Modelo de memoria}

En los sistemas persistentes ortogonales, el modelo de memoria seguido es un
gran espacio de direcciones totalmente cont\'inuo (\cite{Atkinson&Morrison, 1995}). No existe, adem\'as, como se
puede deducir, diferenciaci\'on entre memoria principal y memoria secundaria.

Este modelo de memoria tiene un grave inconveniente: si por alguna raz\'on, un
objeto en memoria se corrompe (o simplemente, se introduce un objeto "malicioso"
en el sistema), \'este est\'a completamente desprotegido, puesto que toda el
espacio de direcciones es accesible. As\'i, un fallo en un puntero podr\'ia provocar
que todos los objetos relacionados con el objeto que produce el fallo se
corrompieran. En un caso a\'un peor, ser\'ia te\'oricamente posible que la
corrupci\'on llegara a alcanzar a todo el sistema.

Los sistemas que siguen este modelo resuelven este problema empleando los lenguajes
llamados "seguros respecto al tipo" (\cite{Atkinson&Morrison, 1995}). As\'i, el usuario, al contrario que
sucede en lenguajes como C o C++, no tiene acceso a punteros que puedan
manejar a su antojo, sino a "referencias" (siguiendo la terminolog\'ia Java \cite{Naughton, 1996})
que son inmodificables por el usuario.

En el modelo basado en contenedores, el espacio de direcciones est\'a totalmente
particionado, empleando de nuevo los mismos contenedores. Cuando un usuario
trabaja en un contenedor, el espacio de direcciones al que tiene acceso es aquel
abarcado por el propio contenedor, y no m\'as. \'esto permite el empleo de lenguajes
no seguros respecto al tipo, como C++. De esta forma, de existir alg\'un
error, \'este queda enmarcado dentro de un espacio limitado, pudiendo corromper,
llegado el caso, todos los objetos en el contenedor, pero no m\'as que \'estos.

Existe una excepci\'on a este modelo totalmente particionado. Ser\'ia imposible
trabajar de forma efectiva en este modelo si todo aquello que el usuario necesita
para realizar una tarea s\'olo puede estar en un determinado contenedor. As\'i,
se permite desde el contenedor emplear objetos en otros contenedores, pero s\'olo
en modo de lectura (\cite{Garc�a Perez-Schofield, 2002}).

\subsection{Funcionamiento}

En esta secci\'on se muestra un ejemplo real, si bien muy simple, de un programa
que lee unos n\'umeros de un fichero, tal y como se ve en el punto siguiente. Las
contrapartidas persistentes se encuentran en los siguientes puntos. N\'otese que
en un sistema persistente los ficheros no existen, puesto que no son necesarios:
los objetos (en este caso, una lista de n\'umeros simples) se almacenan directamente.
\subsubsection{ISO C++}
La versi\'on de este programa en C++ est\'andar es simple. 
El fichero es abierto,
y, a medida que los n\'umeros son le\'idos del mismo, \'estos son mostrados en la
consola.
\begin{verbatim}
int main(void) {
      int x;
      FILE *f  = fopen("datos.dat");
      if (f!=NULL) {
            do {
                  fread(&x, sizeof(int), 1, f);
                  cout << x << endl;
            } while(!feof(f));
            fclose(f);
      } else cerr << "Error de E/S" << endl;
}
\end{verbatim}
\subsubsection{PJama}
PJama (\cite{Atkinson&Jordan, 2000}) es un sistema persistente basado en Java. Existen algunas
clases a\~nadidas que facilitan la comunicaci\'on con el almacenamiento persistente,
como \verb"PJavaStore". Los objetos se recuperan por su nombre, ya que no existe ning\'un
tipo de particionamiento visible para el usuario.

\begin{verbatim}
public class ejemplo {
  public static void main(void) {
      PJavaStore pjs = PJavaStore.getStore();
      int [] vectnum = (int []) pjs.getPRoot("vect_ejemplo");
      for (int j = 0; j < vectnum.size; ++j) 
            System.out.println(vectnum[j]);
  }
}
\end{verbatim}
\subsubsection{Barbados}
En Barbados, no es necesario crear ninguna clase especial o funci\'on especial
como punto de entrada, por lo que se ha elegido una funci\'on simple como
la siguiente. El vector (un vector primitivo de C) es le\'ido de un contenedor
donde est\'a almacenado y es mostrado en la consola.

\begin{verbatim}
void ponVectorEjemplo(void) {
      int *vectnum = /vector/vect_num;
      while (*vectnum != 0) 
            cout << *(vectnum++) << endl;
}
\end{verbatim}
El proceso que ser\'ia necesario seguir para poder crear ese vector en Barbados
ser\'ia el siguiente:

\begin{verbatim}
mkdir(vector);
cd(vector);
int *vect_num = new int[3];
vect_num[0] = 1;
vect_num[1] = 2;
vect_num[2] = 0;
cd(..);
\end{verbatim}

Una vez creado, el sistema puede cerrarse con seguridad: la pr\'oxima vez
que se abra el vector habr\'a sido recuperado con normalidad, y la funci\'on
\verb"ponVectorEjemplo()" podr\'a ser ejecutada sin que antes sea necesaria
ning\'un tipo de intervenci\'on por parte del usuario para cargar los n\'umeros.

%%%%%%%%%%%%%%%%%%%%%%%%%%%%% Conclusiones %%%%%%%%%%%%%%%%%%%%%%%%%%%%%%%%%

\section{Conclusiones}
En este art\'iculo se ha presentado la evoluci\'on de los sistemas operativos,
que se puede concluir que en materia de arquitectura de los mismos, esta 
evoluci\'on ha sido casi nula. Tambi\'en se ha introducido el campo de
investigaci\'on sobre persistencia, para dar a entender una posible
alternativa, que no es otra que la de los sistemas operativos orientados a
objetos.

Se han presentado tambi\'en las razones que justifican un nuevo modelo de
persistencia (esto es, simplificaci\'on que conlleva un impacto mayor en
rendimiento), ofreciendo este nuevo modelo como alternativa que podr\'ia dar
pi\'e a la entrada de los SOOO en el trabajo cotidiano. Si bien se ha explicado
que esta entrada est\'a limitada por varios factores de considerable importancia.

%%%%%%%%%%%%%%%%%%%%%%%%%%%%% Referencias %%%%%%%%%%%%%%%%%%%%%%%%%%%%%%%%%%%

\section{Referencias}


\end{document}